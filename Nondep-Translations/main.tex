\documentclass{article}
\usepackage{nmm}
\usepackage{todonotes}
\setuptodonotes{inline}

\title{Some Notes}
\author{Nathan Mull}

\begin{document}

\maketitle

\abstract{
  \todo{Write an abstract.}
}

\section{Perpetual Reductions}

\todo{Define contexts}

\begin{definition}
A \term{(one-step) reduction strategy} is a mapping $F : \terms \to \terms$ such that $F(M) = M$ if $M \in \mathsf{NF}$ and $M \onestepbeta F(M)$ otherwise.
\end{definition}

\todo{Definition of $F$-path.}

\todo{Definition of strongly normalizing term}

\begin{definition}
A reduction strategy $F$ is \term{perpetual} if, for all expressions $M$ which are not strongly normalizing, the expression $F(M)$ is not strongly normalizing.
In particular, the $F$-path of $M$ is infinite.
\end{definition}

\begin{definition}
  The \term{left-most redex} of an expression $M$ which is not in normal form is defined inductively as follows.
  \begin{itemize}
    \item The left-most redex of $\PiT x A B$ is the left-most redex if $A$ if such a redex exists, and otherwise is the left-most redex of $B$.
    \item The left-most redex of $\lambdaT x A B$ is the left-most redex of $A$ if such a redex exists, and otherwise is the left-most redex of $B$.
    \item The left-most redex of $(\lambdaT x A M) N$ is the left-most redex of $A$ if such a redex exists, and otherwise is the redex $(\lambdaT x A M) N$ itself.
    \item The left-most redex of $M N$ (where $MN$ itself is not a redex) is the left most redex of $M$ if such a redex exists and otherwise is the left most redex of $N$.
  \end{itemize}
\end{definition}

Note that for this notion of left-most redex, we have that if $(\lambdaT x A P)Q$ is a left-most redex, then $A$ is in normal-form.

\begin{definition}
The \term{left-most reduction strategy}, denoted here by $F_\ell$ is the strategy such that $F_\ell(M)$ is the result of reducing the left-most redex of $M$.
\end{definition}

\begin{definition}
The reduction strategy $F_\infty$ is defined as follows.
Let $M$ be an expression.
If $M \in \mathsf{NF}$, then $F_\infty(M) \df M$
Let $M$ be an expression such that $M = C[(\lambdaT x A P)Q]$ where $(\lambdaT x A P) Q$ is the left-most redex of $M$.
\begin{displaymath}
F_\infty(M) \df
\begin{cases}
  C[(\lambdaT x A P)F_\infty(Q)] & \text{$(\lambdaT x A P)Q \in \mathsf{KR}$ and $Q \not \in \mathsf{NF}$} \\
  C[\sub P Q x] & \ow. \\
\end{cases}
\end{displaymath}
\end{definition}

\begin{lemma}
The reduction strategy $F_\infty$ is perpetual.
\end{lemma}

\begin{proof}
Let $M$ an expression which is not strongly normalizing and let $R$ denote the left-most redex $(\lambdaT x A P)Q$ of $M$.
Let $C$ be the context such that $M = C[R]$.
Consider an infinite reduction sequence
\begin{displaymath}
    M \onestepbeta M_1 \onestepbeta M_2 \onestepbeta \dots
\end{displaymath}
First suppose that $R$ is an $I$-redex and $F_\infty(M) = C[\sub P Q x]$.
If the above reduction contains only reduction of non-left-most redexes of $M$, then we can check each case.
\todo{Check each case.}
Otherwise, there is some $k$ such that $M_k \onestepbeta M_{k + 1}$ by reducing the left-most redex of $M_k$.
Then we can follow along up to that point and then continue.
\todo{Write this more carefully.}

Next suppose that $R \in \mathsf{KR}$.
If $Q \in \mathsf{NF}$, then $F_\infty(M) = C[P]$ and since $Q \not \in \mathsf{SN}$, it must be that $C[P] \not \in \mathsf{SN}$ by \autoref{}.
\todo{Write this lemma.}
So suppose that $Q \not \in \mathsf{NF}$.
If $Q \in \mathsf{SN}$, then by \autoref{}, this holds.
\todo{Use lemma!}
Otherwise, we have $F_\infty(Q) \not \in \mathsf{SN}$ by the inductive hypothesis, which implies $F_\infty(M) \not \in \mathsf{SN}$.
\end{proof}

Thus, we have that $F_\infty$ is a perpetual reduction.
We can then use this to prove that if $M \in \mathsf{SN}$, then $\tau_i(M) \in \mathsf{WN}$.
The basic idea is that if $M \in \mathsf{SN}$, then the $F_\infty$-path of $M$ is finite, but if we can simulate this path in the thunkified term and look at the structure of the final term. It will be the same as $M$ up to some erasures, And the erasures only remove terms that are already in normal form.
So it strong normalization can be preserved.

Next we need to show that weak normalization is preserved. This means instead simulating the left-most reduction strategy, which means 

\section{The Translation}

\begin{definition}
For $i \in \dran 2 n$, let $\bot_i$ denote $s_{i - 1}$ and let $\bot_1$ be a distinguished variable.
Also let $\Delta_1$ and $\Delta_2$ denote the context $(\bot_1 : s_1)$ and let $\Delta_i$ denote the empty context in all other cases.\footnote{We define $\Delta_2$ in this way because we want that in all contexts $\Delta_j$ where $j \geq 2$, the expression $\bot_{j - 1}$ is derivable. This will play a similar role to the variable $I$ included in contexts by Barthe \etal (Definition 3.2, \cite{barthe-et-al-2001}).}
\end{definition}

\begin{definition}
Define the families of functions
\begin{displaymath}
  \{\rho_i : \terms_{\geq i} \to \terms \}_{i \in \upto n}
  \qquad \text{and} \qquad
  \{\rho_i' : \terms_{\geq i} \to \terms\}_{i \in \upto n}
\end{displaymath}
simultaneously as follows.
\begin{align*}
  \rho_i(s_j) &\df s_j \qquad \text{(where $j \geq i - 1$)} \\
  \rho_i(\svar j x) &\df \svar j x \qquad \text{(where $j \geq i + 1$)} \\
  \rho_i(\PiT {\svar j x} A B) &\df
    \PiT {\svar j x} {\rho_i'(A)} {\rho_i(B)} \\
  \rho_i(\lambdaT {\svar j x} A M) &\df
    \lambdaT {\svar j x} {\rho_i(A)} {\rho_i(M)} \\
  \rho_i(M N) &\df \rho_i(M) \rho_i(N) \\
  \rho_i'(A) &\df
  \begin{cases}
    \bot_i \to \rho_i(A) & \deg(A) = i \\
    \rho_i(A) & \ow
  \end{cases}
\end{align*}
The functions in $\{\rho_i'\}_{i \in \upto n}$ are extended to contexts as follows.
\begin{align*}
\rho_i'(\emptyContext) &\df (\statement {\bot_i} {s_i}) \\
\rho_i'(\Gamma, \statement {\svar j x} A) &\df
\begin{cases}
  \rho_i'(\Gamma), \statement {\avar s x} {\rho_i'(A)} & j \geq i \\
  \rho_i'(\Gamma) & \ow. \\
\end{cases}
\end{align*}
\end{definition}

\begin{lemma}
\label{lem:rho-commutes-sub}
($\rho_i$ and $\rho_i'$ commute with substitution) For index $i$, variable $\svar j x$ and expressions $M$ and $N$ where $\deg(N) = j - 1$ the following hold.
\begin{itemize}
\item $\rho_i(\sub M N {\svar j x}) = \sub {\rho_i(M)} {\rho_i(N)} {\svar j x}$
\item $\rho_i'(\sub M N {\svar j x}) = \sub {\rho_i'(M)} {\rho_i(N)} {\svar j x}$
\end{itemize}
\end{lemma}

\begin{lemma}
\label{lem:rho-beta}
($\rho_i$ and $\rho_i'$ preserve $\beta$-reductions)
For index $i$ and derivable expressions $M$ and $N$,
\begin{itemize}
\item if $M \onestepbeta N$ then $\rho_i(M) \onestepbeta \rho_i(N)$ and $\rho_i'(M) \onestepbeta \rho_i'(N)$;
\item if $M \eqbeta N$ then $\rho_i(M) \eqbeta \rho_i(N)$ and $\rho_i'(M) \eqbeta \rho_i'(N)$.
\end{itemize}
\end{lemma}

\begin{definition}
For any $n$-tiered pure type system $\lS$ define $\lS^\bot$ to be the $(n + 1)$-tiered system specified by
\begin{align*}
  \rules_{\lS^\bot} &\df \rules_{\lS} \cup \setcomp {(s_j, s_i)} {j \geq i}
\end{align*}
\end{definition}

\begin{lemma}
\label{lem:rho-pres-type}
($\rho_i$ and $\rho_i'$ preserve typability) Let $\lS$ be a non-dependent tiered pure type systems.
For index $i$, context $\Gamma$, and expressions $M$ and $A$, if $s_i$ is negatable then the following hold.
\begin{enumerate}
\item\label{item:rho-preserves-types-i} If $\judgment[\lS] \Gamma M A$ and $\deg M \geq i$ then $\judgment[\lS^\bot] {\rho_i'(\Gamma)} {\rho_i(M)} {\rho_i(A)}$.
\item\label{item:rho-preserves-types-ii} If $\judgment[\lS] \Gamma A {s_j}$ and $\deg A \geq i$ then $\judgment[\lS^\bot]{\rho_i'(\Gamma)} {\rho_i'(A)} {s_j}$.
\end{enumerate}
\end{lemma}

\begin{proof}
Simultaneously by induction on the structure of derivations.

\indcase{Axiom}

\indcase{Variable Introduction}

\indcase{Weakening}

\indcase{Product Type Formation}

\indcase{Abstraction}

\indcase{Application}

\indcase{Conversion}
\end{proof}

\begin{lemma}
If $M \in \sn$, then $\tau_i(M) \in \wn$.
\end{lemma}

\begin{proof}
Since $M$ is strongly normalizing, the $F_\infty$-path of $M$ is finite.
We can follow along this path.
\end{proof}

\begin{definition}
The reduction strategy $F_\tau$ is defined as follows.
Let $M$ be a term with a redex $(\lambdaT x A M\langle x, \cdot \rangle)N$.
\end{definition}

I know realize this is no different than the last result.
Because we're reducing something fully on $K$-redexes, which is sort of equivalent to the case of the perpetual reduction.
Right, this is the perpetual reduction.

\begin{lemma}
If $M \in \wn$ and $M$ is derivable in $\lS$, then $\tau_i(M) \in \wn$.
\end{lemma}

\begin{proof}
Consider the left-most reduction sequence of $M$, which is guaranteed to be finite since $M \in \wn$.
The goal is to look at the following sequence.
\end{proof}

\bibliographystyle{plain}
\bibliography{references.bib}

\end{document}s
